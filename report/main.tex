\documentclass[12pt,a4paper]{article}

\usepackage[utf8]{inputenc}
\usepackage{hyperref}
\usepackage{parskip}
\usepackage{amsfonts,amsmath}

\hypersetup{
    colorlinks=true,
    linkcolor=blue,
    urlcolor=blue,
}

\title{Security 1 - Mandatory Hand-in 1}
\date{\today}
\author{Sam Al-Sapti (sals@itu.dk)}


\begin{document}
    \maketitle

    \section*{Setup}

    I have decided to implement the assignment in Python, as it is a simple
    language, yet powerful enough for the purposes of the assignment. The
    program can be run by executing \texttt{python3 main.py} or
    \texttt{./main.py}.
    
    \section*{Part 1}

    I start by defining variables \texttt{g} and \texttt{p}, and then define
    \texttt{group} as a list from $0$ to $p - 1$, representing the cyclic
    group $(\mathbb{Z}^*_p, \cdot)$.

    I then define the function \texttt{encrypt(pk, m)}, that uses the ElGamal
    encryption algorithm to encrypt a plaintext message \texttt{m} with a
    public key \texttt{pk}, and returns a tuple \texttt{(c1, c2)}. The
    algorithm goes as follows:

    \begin{enumerate}
        \item Select a random $r \in \mathbb{Z}^*_p$.
        \item Compute $c_1 = g^r \bmod p$.
        \item Compute $c_2 = {pk}^r \cdot m \bmod p$.
        \item Return the ciphertext $(c_1, c_2)$.
    \end{enumerate}

    With this function, Alice encrypts the message \texttt{2000} as an
    integer, using Bob's public key.

    \section*{Part 2}

    For this part, I use a brute-force attack to find Bob's private key. I
    iterate over \texttt{group} and compare the result of $g^{sk} \bmod p$
    with Bob's public key, which is known as $2227$, for each \texttt{sk} in
    \texttt{group}. When an \texttt{sk} is found such that $g^{sk} \bmod p =
    2227$, the iteration breaks and \texttt{sk} is used as Bob's private key
    to decrypt Alice's ciphertext, using the function \texttt{decrypt(sk, c)}.
    Eve finds Bob's private key to be $66$.

    The function takes a private key \texttt{sk} and the ciphertext as the
    tuple \texttt{c = (c1, c2)}, and follows the following decryption
    algorithm:

    \begin{enumerate}
        \item Compute $s = {c_1}^{sk} \bmod p$.
        \item Compute $m = c_2 \cdot s^{-1} \bmod p$.
        \item Return $m$.
    \end{enumerate}

    Due to type conversion issues occurring when dividing integers in Python,
    I have utilized Fermat's little
    theorem\footnote{\url{https://en.wikipedia.org/wiki/Fermat\%27s\_little\_theorem}}
    to instead compute $m = c_2 \cdot s^{p - 2} \bmod p$ in step 2.

    Using Bob's private key that Eve has found, Alice's ciphertext is
    successfully decrypted, yielding the original plaintext \texttt{2000}.

    \section*{Part 3}

    In this part of the assignment, Mallory is assumed to know that the
    original plaintext reads \texttt{2000}.

    The ciphertext contains $(c_1, c_2)$, where $c_2 = {pk}^r \cdot m \bmod
    p$. By choosing an appropriate modifier $d$, Mallory can successfully
    compute a new $c_2'$, as long as $d \cdot m < p$ holds. Since Mallory
    knows the plaintext to be \texttt{2000}, he can choose $d = 3$ to compute
    $c_2' = {pk}^r \cdot 3m \bmod p$ and modify the ciphertext to $(c_1,
    c_2')$. Since $m = 2000$ and $3m = 6000$, Bob will get the plaintext
    \texttt{6000} when he decrypts the modified ciphertext.

\end{document}